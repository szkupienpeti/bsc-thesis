% TeXstudio spellcheck 2020-12-10 16:31

%----------------------------------------------------------------------------
\chapter{\bevezetes}
%----------------------------------------------------------------------------
%\vspace{-1cm}
Biztonságkritikus informatikai rendszerek esetében kiemelt fontosságú, hogy megbizonyosodhassunk arról, megfelel-e az elkészített rendszer azoknak a követelményeknek, amelyeket eredetileg támasztottak vele szemben. Ennek ellenkezője ugyanis beláthatatlan következményekkel járhat: gondoljunk például atomerőművek vezérlésére, autóbuszok fékjére, repülők hajtóművére. Egy elkészült szoftverrendszer ellenőrzésének egyik módszere a tesztelés: annak vizsgálata, hogy a rendszer bizonyos bemenetekre (tesztesetekre) hogyan viselkedik, milyen állapotba kerül (\emph{white box test}), milyen kimenetet produkál (\emph{black box test}). Noha vannak ökölszabályok (pl. ekvivalencia-osztályok, határérték-analízis stb.) arra, hogy egy rendszer tesztelésekor milyen elvek mentén érdemes tesztbemeneteket választani ahhoz, hogy azok minél inkább lefedjék a rendszer lehetséges működéseit, ezek a kézzel megválasztott tesztesetek legfeljebb valószínűsíthetik a helyes működést, de nem alkalmasak arra, hogy egy rendszerről matematikai precizitással \emph{bizonyítsanak} valamit.

A megoldást a rendszer és a követelmények formalizálása jelenti, ami lehetővé teszi a modellek ellenőrzését formális módszerekkel. Ezek a módszerek matematikai alapokon nyugszanak, és a rendszer összes viselkedésének figyelembe vételével bizonyosodnak meg egy követelmény teljesüléséről vagy annak hiányáról. Nem elég azonban önmagában, ha egy algoritmus azt a kimenetet adja, hogy a rendszerünk nem teljesít egy követelményt, az is hasznos, ha ilyenkor megkapjuk a rendszer egy konkrét lefutását, amely egy olyan állapotba vezet, ahol a követelmény ténylegesen sérül – megkönnyítve ezzel a hiba javítását.

Nemcsak a követelmények sérülésekor hasznos a konkrét lefutások generálása: \emph{modellalapú tesztgenerálás} esetén a modellellenőrzés során feltárt teljes állapottér felhasználásával generálunk olyan tesztkészletet, amely lefedi a rendszer minél több (bizonyos metrika szerint akár összes releváns) működését. Ennek a jelentőségét az adja, hogy hiába tudjuk egy formális modellről biztosan, hogy teljesít egy követelményt, az implementáció során is kerülhetnek hibák a rendszerbe, amelyeket már valóban csak tesztekkel (a rendszer konkrét futtatásával) lehet kiszűrni. Éppen ezért hasznos, ha az elméletben helyes rendszer tényleges megvalósítása után is le tudjuk ellenőrizni, hogy teljesülnek-e a követelmények -- ha nem, akkor valójában nem a formális modell által leírt rendszert valósítottuk meg.

Időzített rendszerek esetén a teljes állapottér gyakran végtelen, így annak (minél nagyobb arányú) bejárása csak absztrakció segítségével lehetséges. Az állapotok absztrahálásával egy olyan absztrakt állapottérre egyszerűsítjük a feladatot, amely már véges, így kezelhető. Egy absztrakt állapottérben egy út azonban nem egy konkrét viselkedést (tesztesetet) ír le, hanem akár végtelen sokat.

Ha a modellellenőrző algoritmus talál egy utat az absztrakt állapottérben, az út menti állapotváltozásokhoz még konkrét időzítést is generálnunk kell. Ennek az időzítésnek pedig olyannak kell lennie, hogy az általa kapott konkrét út (teszteset) valóban abba az absztrakt útba (tesztesetbe) tartozzon, amit a modellellenőrző algoritmus eredetileg talált. Ezt az biztosítja, ha a konkrét útra teljesül minden olyan feltétel, amivel a modellellenőrző algoritmus az absztrakt utat specifikálta.

Az időzített automaták területének megkerülhetetlen eszköze az UPPAAL \cite{UPPAAL}. Az első kiadás 1995-ös dátuma is mutatja, hogy a probléma valós és régóta fennáll. Az UPPAAL-ban időzített automaták hálózatait modellezhetjük, vagyis nem pusztán az automatákat, de az azok közti kommunikációt is. Ezen rendszerekre megfogalmazhatunk feltételeket, amelyek teljesülését az UPPAAL modellellenőrzője ellenőrzi, azok sérülése esetén pedig ún. \emph{diagnostic trace}-t ad vissza, amely a rendszerünk egy olyan konkrét lefutása, amely abba az állapotba vezet, ahol a feltétel sérül. Ezeket a trace-eket szimulálhatjuk is. Az is megadható, hogy egy feltétel sérülése esetén milyen trace-t szeretnénk kapni: választhatjuk a \emph{legrövidebbet} (lépésszámban) vagy a \emph{leggyorsabbat} (időben) is.

A Theta \cite{Theta} modellellenőrző keretrendszert a Budapesti Műszaki és Gazdaságtudományi Egyetem Méréstechnika és Információs Rendszerek Tanszékén fejlesztik. Előnye, hogy több formalizmuson is működik, és felépítéséből adódóan könnyedén bővíthető továbbiakkal is. A modellellenőrző magja absztrakciót használ, amely hatékony működést eredményez.

Szakdolgozatomban a Theta modellellenőrző keretrendszer kiegészítését mutatom be: egy időzített automata absztrakt reprezentációjából olyan konkrét tesztesetek (időzített lépéssorozatok) generálását, amelyek \emph{lefedik a modell összes vezérlési helyét}.

A tesztesetek milyenségét illetően az UPPAAL-ban megismert célok (legrövidebb, leggyorsabb) itt is relevánsak, azonban nemcsak a tesztesetekre, hanem a teljes tesztkészletre vonatkozóan is értelmezhetők hasonló metrikák.

A fejlesztés távlatibb célja egy olyan folyamat támogatása, amelynek bemenete egy időzített mérnöki modell (pl. egy SysML állapotgép Magic Draw-ból\footnote{\url{https://www.nomagic.com/products/magicdraw}}), kimenete pedig egy (adott követelményeknek megfelelő) tesztkészlet. Ennek a folyamatnak a köztes lépése a szintén a Budapesti Műszaki és Gazdaságtudományi Egyetem Méréstechnika és Információs Rendszerek Tanszékén fejlesztett Gamma \cite{Gamma} eszköz segítségével történik, amely magas szintű mérnöki modelleket képes alacsonyabb szintű formális reprezentációkra (esetünkben pl. időzített automaták hálózatára) leképezni. A Gamma által előállított időzített automatákon futna tehát a Theta modellellenőrzője és tesztgenerálása. Ez a komplex tesztgenerálási folyamat komoly előrelépés lenne az időzített biztonságkritikus rendszerek fejlesztésében.

Szakdolgozatom \ref{hatterismeretek}. fejezetében áttekintem a témához kapcsolódó háttérismereteket: a modellek és követelmények formalizálását, valamint a modellellenőrzés elméletét. Bemutatom az időzített modellek formális leírását, annak szintaktikáját és szemantikáját, valamint, hogy milyen absztrakcióval tehető végessé a kezdetben végtelen állapottér, milyen adatstruktúrával írható le a véges, absztrakt modell. Áttekintem a kényszerkielégíthetőségi problémák leírását és ezek megoldását, majd bemutatom a Theta modellellenőrző keretrendszert. A \ref{main-elmelet}. fejezetben formalizálom a feladatot, és részletesen bemutatom az általam megvalósított tesztgeneráló algoritmust, majd a \ref{main-gyakorlat}. fejezetben annak implementációját is áttekintem. Az \ref{esettanulmany}. fejezetben egy esettanulmányon keresztül bemutatom a teljes tesztgenerálási folyamatot, majd áttekintem az implementációm mérési eredményeit. Végül a \ref{osszefoglalas}. fejezetben összefoglalom a munkámat és sorra veszem a megoldásom továbbfejlesztési lehetőségeit.