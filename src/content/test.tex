%----------------------------------------------------------------------------
\chapter{\bevezetes}
%----------------------------------------------------------------------------
%\vspace{-1cm}
A bevezető tartalmazza a diplomaterv-kiírás elemzését, történelmi előzményeit, a feladat indokoltságát (a motiváció leírását), az eddigi megoldásokat, és ennek tükrében a hallgató megoldásának összefoglalását.

A bevezető szokás szerint a diplomaterv felépítésével záródik, azaz annak rövid leírásával, hogy melyik fejezet mivel foglalkozik \cite{Candy86}.

Ez egy képlet (lásd \ref{fig:my_label-1}): $\mathit{függvény}(x) = \int_{x=0}^{x=100} x^2$.

\[
\mathit{függvény}(x) = \int_{x=0}^{x=100} x^2
\]

\begin{equation*}
    \mathit{függvény}(x) = \int_{x=0}^{x=100} x^2
\end{equation*}

\begin{figure}[t]
    \begin{minipage}[b]{0.49\textwidth}
        \centering
        \includegraphics{}
        \vspace{0}
        \subcaption{Al-ábra 1}
        \label{fig:my_label-1}
    \end{minipage}
    \begin{minipage}[b]{0.49\textwidth}
        \centering
        \includegraphics{}
        \vspace{0}
        \subcaption{Al-ábra 2}
        \label{fig:my_label-2}
    \end{minipage}
    \caption{Fő ábra}
    \label{fig:my_label}
\end{figure}

``valami''

$v_kiskocsi = 1$

\newcommand{\aban}[1]{\atold#1+ban{}}
\aban{\ref{fig:idozitett-automata-pelda}}